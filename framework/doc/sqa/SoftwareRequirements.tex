\documentclass{INLreport}
\SANDnum{MOOSE-SRS}
\SANDprintDate{February 2016}
\SANDreleaseType{Unlimited Release}
\SANDauthor{Cody Permann}
\title{Multiphysics Object Oriented Simulation Environment (MOOSE) Software Requirements Specification and Software Design Description}

\usepackage[table]{xcolor}

\usepackage{caption}
\captionsetup{labelsep=space,justification=justified,labelfont=bf,singlelinecheck=off}

% Use for numbering rows of tables.
\newcommand{\TableRowNum}[1]{\thetable.\arabic{#1}\stepcounter{#1}}

\begin{document}
\maketitle

\SANDmain

\section{Introduction}
\subsection{Purpose}
The Multiphysics Object-Oriented Simulation Environment (MOOSE) is a tool for solving complex coupled Multiphysics equations
using the finite element method. MOOSE uses an object-oriented design to abstract data structure management, parallelism,
threading and compiling while providing an easy to use interface targeted at engineers that may not have a lot of
software development experience. MOOSE will require extreme scalability and flexibility when compared to other FEM
frameworks. For instance, MOOSE needs the ability to run extremely complex material models, or even third-party applications
within a parallel simulation without sacrificing parallelism. This capability is in contrast to what is often seen in
commercial packages, where custom material models can limit the parallel scalability, forcing serial runs in the most severe
cases. When comparing high-end capabilities, many MOOSE competitors target modest-sized clusters with just a few thousand
processing cores. MOOSE, however, will be required to routinely executed on much larger clusters with scalability to clusters
available in the top 500 systems. (top500.org). MOOSE will also be targeted at smaller systems such as high-end laptop
computers.

The design goal of MOOSE is to give developers ultimate control over their physical models and applications. Designing new
models or solving completely new classes of problems will be accomplished by writing standard C++ source code within the
framework’s class hierarchy. Scientists and engineers will be free to implement completely new algorithms using pieces of the
framework where possible, and extending the framework’s capabilities where it makes sense to do so. Commercial applications do
not have this capability, and instead opt for either a more rigid parameter system or a limited application-specific metalanguage.

\subsection{Scope}
MOOSE's scope is to provide a set of interfaces for building Finite Element Modeling (FEM) simulations. Abstractions to all
underlying libraries are provided.

Solving coupled problems where competing physical phenomena impact one and other in a significant nonlinear fashion represents a
serious challenge to
several solution strategies. Small perturbations in strongly-coupled parameters often have very large adverse effects on
convergence behavior. These adverse effects are compounded as additional physics are added to a model. To overcome these
challenges, MOOSE employs three distinct yet compatible systems for solving these types of problems.

First, an advanced numerical technique called the Jacobian-Free Newton-Krylov (JFNK) method is employed to solve the most fully-coupled physics
in an accurate, consistent way. An example of this would be the effect of temperature on the expansion or contraction of a
material. While the JFNK numerical method is very effective at solving fully-coupled equations, it can also be computationally
expensive. Plus, not all physical phenomena in a given model are truly coupled to one another. For instance, in a reactor, the
speed of the coolant flow may not have any direct effect on the complex chemical reactions taking place inside the fuel rods.
We call such models ``loosely-coupled''. A robust, scalable system must strike the proper balance between the various modeling
strategies to avoid performing unnecessary computations or incorrectly predicting behavior in situations such as these.

MOOSE’s Multiapp system will allow modelers to group physics into logical categories where MOOSE can solve some groups
fully-coupled and others loosely-coupled. The Multiapp system goes even further by also supporting a “tightly-coupled”
strategy, which falls somewhere between the “fully-coupled” and “loosely-coupled” approaches. Several sets of physics can then
be linked together into logical hierarchies using any one of these coupling strategies, allowing for several potential solution
strategies. For instance, a complex nuclear reactor model might consist of several tightly-coupled systems of fully-coupled
equations.

Finally, MOOSE’s Transfers system ties all of the physics groups contained within the Multiapp system together and allows for
full control over the flow of information among the various groups. This capability bridges physical phenomena from several
different complementary scales simultaneously. When these three MOOSE systems are combined, myriad coupling combinations are
possible. In all cases, the MOOSE framework handles the parallel communication, input, output and execution of the underlying
simulation. By handling these computer science tasks, the MOOSE framework keeps modelers focused on doing research.

MOOSE innovates by building advanced simulation capabilities on top of the very best available software technologies in a way that
makes them widely accessible for innovative research. MOOSE is equally capable of solving small models on common laptops and
the very biggest FEM models ever attempted -- all without any major changes to configuration or source code. Since its
inception, the MOOSE project has focused on both developer and computational efficiency. Improved developer efficiency is
achieved by leveraging existing algorithms and technologies from several leading open-source packages. Additionally, MOOSE uses
several complementary parallel technologies (both the distributed-memory message passing paradigm and shared-memory thread-based
approaches are used) to lay an efficient computational foundation for development. Using existing open technologies in this
manner helps the developers reduce the scope of the project and keeps the size of the MOOSE code base maintainable. This approach
provides users with state-of-the-art finite element and solver technology as a basis for the advanced coupling and solution
strategies mentioned previously.

MOOSE’s developers work openly with other package developers to make sure that cutting-edge
technologies are available through MOOSE, providing researchers with competitive research opportunities. MOOSE maintains a set
of objects that hide parallel interfaces while exposing advanced spatial and temporal coupling algorithms in the framework.
This accessible approach places developmental technology into the hands of scientists and engineers, which can speed the pace of
scientific discovery.

\subsection{Overview}
\subsubsection{User Characteristics}
\begin{itemize}
\item Framework Developers: These are the core developers of the framework. They will be responsible for following
  and enforcing the appropriate software development standards. They will be responsible for designing, implementing
  and maintaining the software.

\item Developers: A Scientist or Engineer that utilizes the framework to build his or her own application. This user
  will typically have a background in modeling and simulation techniques and/or numerical analysis but may only have
  a limited skill-set when it comes to object-oriented coding and the C++ language. This is our primary focus group.
  In many cases these developers will be encouraged to give their code back to the framework maintainers.

\item Analysts: These are users that will run the code and perform various analysis on the simulations they perform.
  These users may interact with developers of the system requesting new features and reporting bugs found and will
  typically make heavy use of the input file format.
\end{itemize}


\subsubsection{Assumptions and Dependencies}
The software should be designed with the fewest possible constraints. Ideally the software should run on a wide
variety of evolving hardware so it should follow well-adopted standards and guidelines. The software should run
on any POSIX compliant system. The software will also make use FEM and numerical libraries that run on POSIX
systems as well. The main interface for the software will be command line based with no assumptions requiring advanced
terminal capabilities such as coloring and line control.

\section{References}

\begin{itemize}
\item ASME NQA 1 2008 with the NQA-1a-2009 addenda, ``Quality Assurance Requirements for Nuclear Facility Applications,'' First Edition, August 31, 2009.
\item ISO/IEC/IEEE 24765:2010(E), ``Systems and software engineering — Vocabulary,'' First Edition, December 15, 2010.
\item LWP 13620, ``Managing Information Technology Assets''
\end{itemize}

\section{Definitions and Acronyms}

\subsection{Definitions}

\begin{itemize}
\item \textit{Baseline}. A specification or product (e.g., project plan, maintenance and operations [M\&O] plan, requirements,
  or design) that has been formally reviewed and agreed upon, that thereafter serves as the basis for use and further development,
  and that can be changed only by using an approved change control process. [ASME NQA-1-2008 with the NQA-1a-2009 addenda edited]

\item \textit{Validation}. Confirmation, through the provision of objective evidence (e.g., acceptance test), that the requirements
  for a specific intended use or application have been fulfilled. [ISO/IEC/IEEE 24765:2010(E) edited]

\item \textit{Verification}. (1) The process of: evaluating a system or component to determine whether the products of a given
  development phase satisfy the conditions imposed at the start of that phase. (2) Formal proof of program correctness (e.g.,
  requirements, design, implementation reviews, system tests). [ISO/IEC/IEEE 24765:2010(E) edited]
\end{itemize}

\subsection{Acronyms}

\begin{tabular}{l l}
  ASME    & American Society of Mechanical Engineers \\
  DOE     & Department of Energy \\
  FEM     & Finite Element Modeling \\
  IEC     & International Electromechanical Commission \\
  IEEE    & Institute of Electrical and Electronics Engineers \\
  INL     & Idaho National Laboratory \\
  ISO     & International Organization for Standardization \\
  IT      & Information Technology \\
  M\&O    & Maintenance and Operations \\
  MOOSE   & Multiphysics Object Oriented Simulation Environment \\
  NQA     & Nuclear Quality Assurance \\
  OOP     & Object Oriented Programming \\
  QA      & Quality Assurance \\
  V\&V    & Verification and Validation \\
\end{tabular}


\section{System Requirements}

\begin{table}[!htbp]
  \caption{Minimum Requirements.\label{tab:req}}
  \begin{tabular}{|p{12cm}|}
    \rowcolor{gray}
    Minimum Requirements \\
    \begin{itemize}
    \item A POSIX compliant Unix including the two most recent versions of OS X and most current versions of Linux.
    \item RAM: 4 GB for optimized compilation (8 GB for debug compilation), 2 GB per core execution
    \item Disk: 100 GB
    \item Compilers: GCC, Clang, or Intel
    \item Python 2.6+
    \item Git
    \end{itemize} \\
    \hline
  \end{tabular}
\end{table}


\subsection{Functional Requirements}

\begin{table}[!htbp]
  \caption{Function Name: Core Framework, I/O and Execution Control.\label{tab:core}}
  \begin{tabular}{|l|p{12cm}|}
    \rowcolor{gray}
    Number & Requirement Description \\
    F1.10 & The system shall allow support for user-defined object for controlling the execution stages of a simulation. \\ \hline
    F1.20 & The system shall allow for user-defined time step selection for simulations using a time-based execution scheme. \\ \hline
    F1.30 & The system shall allow for user-defined time step integration schemes for simulations using a time-based execution scheme. \\ \hline
    F1.40 & The system shall support user-defined matrix preconditioning that may be applied during the solve stages. \\ \hline
    F1.50 & The system should support a programmatic method for building up the necessary objects necessary for a simulation. \\ \hline
    F1.60 & The system shall provide the ability to resume a previous simulation due to fault or intentional termination. \\ \hline
    F1.70 & The system shall allow for user-defined output types for simulation data. \\ \hline
    F1.80 & The system shall provide a method of providing improved initial guesses (also known as "predictions") for the solution at subsequent time steps. \\ \hline
    F1.90 & The system shall support the creation of constraints relating otherwise independent degrees of freedom. \\ \hline
  \end{tabular}
\end{table}

\begin{table}[!htbp]
  \caption{Function Name: Mesh and Mesh Refinement.\label{tab:core}}
  \begin{tabular}{|l|p{12cm}|}
    \rowcolor{gray}
    Number & Requirement Description \\ \hline
    F2.10 & The system shall support user-defined mesh creation. \\ \hline
    F2.20 & The system shall support user-defined mesh modification. \\ \hline
    F2.30 & The system shall support user-defined mesh partitioning. \\ \hline
    F2.40 & The system shall allow for the computation of user-defined quantities for the purpose of mesh refinement. \\ \hline
    F2.50 & The system shall allow for the computation of user-defined element markings for the purpose of mesh refinement. \\ \hline
  \end{tabular}
\end{table}


\begin{table}[!htbp]
  \caption{Function Name: Finite Element Assembly.\label{tab:fem}}
  \begin{tabular}{|l|p{12cm}|}
    \rowcolor{gray}
    Number & Requirement Description \\ \hline
    F3.10 & The system shall allow for the creation of many independently discretized nonlinear field variables. \\ \hline
    F3.20 & The system shall provide a mechanism for setting values of all implicit field variables prior to the start of a simulation. \\ \hline
    F3.30 & The system shall allow the input of ``weak-form'' operators for computing element-wise volume integrals. \\ \hline
    F3.40 & The system shall allow the input of ``weak-form'' operators for computing side integrals along individual boundaries. \\ \hline
    F3.50 & The system shall allow the input of ``Dirac delta functions'' contributions into simulation. \\ \hline
    F3.60 & The system shall support weak-form contributions from internal boundaries necessary for ``Discontinuous Galerkin'' approaches. \\ \hline
  \end{tabular}
\end{table}

\begin{table}[!htbp]
  \caption{Function Name: Materials System.\label{tab:mat}}
  \begin{tabular}{|l|p{12cm}|}
    \rowcolor{gray}
    Number & Requirement Description \\ \hline
    F4.10 & The system shall allow for the creation of objects that represent physical material properties based on physical quantities. \\ \hline
  \end{tabular}
\end{table}


\begin{table}[!htbp]
  \caption{Function Name: Auxiliary Calculations.\label{tab:aux}}
  \begin{tabular}{|l|p{12cm}|}
    \rowcolor{gray}
    Number & Requirement Description \\ \hline
    F5.10 & The system shall allow for the creation of many independently discretized and explicitly calculated ``field'' variables. \\ \hline
    F5.20 & The system shall provide a mechanism for setting values of all explicit field variables prior to the start of a simulation. \\ \hline
    F5.30 & The system shall allow for the creation of operators for explicitly calculating spatial quantities. \\ \hline
  \end{tabular}
\end{table}

\begin{table}[!htbp]
  \caption{Function Name: User-defined logic, functions and post-processing.\label{tab:pp}}
  \begin{tabular}{|l|p{12cm}|}
    \rowcolor{gray}
    Number & Requirement Description \\ \hline
    F6.10 & The system shall provide interfaces for computing single value domain calculations based on other system information or field variables. \\ \hline
    F6.20 & The system shall allow for the creation and use of arbitrary functions of space and time for use by other objects in the simulation. \\ \hline
    F6.30 & The system shall provide interfaces for computing multiple value domain calculations based on other system information or field variables. \\ \hline
    F6.40 & The system shall support additional user-defined algorithms that run during the simulation and may be used by other system objects. \\ \hline
    F6.50 & The system shall support interfaces for handling mathematical operations among geometric entities. \\ \hline
  \end{tabular}
\end{table}

\begin{table}[!htbp]
  \caption{Function Name: Multiple Application Execution and Communication.\label{tab:multiapp}}
  \begin{tabular}{|l|p{12cm}|}
    \rowcolor{gray}
    Number & Requirement Description \\ \hline
    F7.10 & The system shall support loose coupling mechanisms and execution of multiple applications in a coordinated fashion. \\ \hline
    F7.20 & The system shall support solution transfer between multiple mesh structures. \\ \hline
  \end{tabular}
\end{table}

\begin{table}[!htbp]
  \caption{Function Name: Control Logic.\label{tab:cl}}
  \begin{tabular}{|l|p{12cm}|}
    \rowcolor{gray}
    Number & Requirement Description \\ \hline
    F8.10 & The system should support a way to inspect variable values and change simulation parameters during a simulation. \\ \hline
  \end{tabular}
\end{table}


\subsection{Usability Requirements}

\begin{table}[!htbp]
  \caption{Core Framework\label{tab:usabilty_core}}
  \begin{tabular}{|l|p{12cm}|}
    \rowcolor{gray}
    Number & Requirement Description \\ \hline
    U1.10 & The system will be command-line and input file driven. \\ \hline
    U1.20 & The system shall return usage messages when unidentified arguments or incorrectly used arguments are passed. \\ \hline
    U1.30 & The system shall provide diagnostics when the input file fails to parse, or the format is incorrect. \\ \hline
    U1.40 & The system will provide on screen information about the simulation and performance characteristics of the solves under normal operating conditions. \\ \hline
  \end{tabular}
\end{table}

\clearpage

    \subsection{Performance Requirements}

\begin{table}[!htbp]
  \caption{Core Framework\label{tab:perf_core}}
  \begin{tabular}{|l|p{12cm}|}
    \rowcolor{gray}
    Number & Requirement Description \\ \hline
    P1.10 & The system will support multi-process distributed memory execution. \\ \hline
    P1.20 & The system will support multi-process shared memory execution. \\ \hline
    P1.30 & The system will support execution on Unix-based laptops. \\ \hline
    P1.40 & The system will support execution on Unix-based workstation systems. \\ \hline
    P1.50 & The system will support execution on large Unix-based cluster systems. \\ \hline
  \end{tabular}
\end{table}

\clearpage

\subsection{System Interfaces}

\begin{table}[!htbp]
  \caption{Core Framework\label{tab:perf_core}}
  \begin{tabular}{|l|p{12cm}|}
    \rowcolor{gray}
    Number & Requirement Description \\ \hline
    S1.10 & The system shall support POSIX compliant systems. \\ \hline
    S1.20 & The system shall support the Message Passing Interface (MPI) standard. \\ \hline
    S1.30 & The system shall support POSIX ``pthreads''. \\ \hline
    S1.40 & The system shall support Intel Threaded Building Blocks (TBB) interface. \\ \hline
    S1.50 & The system shall support the OpenMP threading interface. \\ \hline
  \end{tabular}
\end{table}

\clearpage

\subsection{System Operations}

\subsubsection{Human System Integration Requirements}
The command line interface shall support the ability to toggle any supported coloring schemes on or off pursuant to section 508 of the Rehabilitation Act of 1973.

\subsubsection{Maintainability}
\label{Maintainability}
\begin{enumerate}
\item The latest working version (defined as the version that passes all tests in the current regression test suite)
      shall be publicly available at all times through the repository host provider.
\item Flaws identified in the system shall be reported and tracked in a ticket or issue based system. The technical lead
      will determine the severity and priority of all reported issues and assign resources at his or her discretion to
      resolve identified issues.
\item The software maintainers will entertain all proposed changes to the system in a timely manner (within two business days).
\item The core framework in its entirety will be made publicly available under the LGPL version 2.0 license.
\end{enumerate}

\subsubsection{Reliability}
The regression test suite will cover at least 80\% of all lines of code at all times. Known regressions will be recorded
and tracked (see \ref{Maintainability}) to an independent and satisfactory resolution.

\subsection{Information Management}
The core framework in its entirety will be made publicly available on an appropriate repository hosting site. Backups and
security services will be provided by the hosting service.

\section{Verification}

The regression test suite will employ several verification tests using comparison against known analytical solutions, the method of
manufactured solutions, and convergence rate analysis.


%%%%%%%%%%%%%%%%%%%%%%%%%%%%%%%%%%%%%%%%%%%%%%
% Software Design Description
%%%%%%%%%%%%%%%%%%%%%%%%%%%%%%%%%%%%%%%%%%%%%%
\section{Software Design Description}

\subsection{Introduction}
Frameworks are a software development construct aiming to simplify the creation of specific classes of applications
through abstraction of low-level details. The main object of creating a framework is to provide an interface to
application developers that saves time and provides advanced capabilities not attainable otherwise. The Multiphysics
Object Oriented Simulation Environment (MOOSE), mission is just that: provide a framework for engineers and scientists
to build state-of-the-art, computationally scalable finite element based simulation tools.

MOOSE was conceived with one major objective: to be as easy and straightforward to use by scientists and engineers as possible.
MOOSE is meant to be approachable by non-computational scientists who have systems of partial differential equations (PDEs)
they need to solve. Every single aspect of MOOSE was driven by this singular principle from the build system to the API to the
software development cycle.  At every turn, decisions were made to enable this class of users to be successful with the framework.
The pursuit of this goal has led to many of the unique features of MOOSE:

\begin{itemize}
  \item A streamlined build system
  \item An API aimed at extensible
  \item Straightforward APIs providing sensible default information
  \item Integrated, automatic, and rigorous testing
  \item Rapid, continuous integration development cycle
  \item Codified, rigorous path for contributing
  \item Applications are modular and composable
\end{itemize}

Each of these characteristics is meant to build trust in the framework by those attempting to use it. For instance,
the build system is the first thing potential framework users come into contact with when they download a new
software framework. Onerous dependency issues, complicated, hard to follow instructions or build failure can all
result in a user passing on the platform. Ultimately, the decision to utilize a framework comes down to whether or
not you trust the code in the framework and those developing it to be able to support your desired use-case.
No matter the technical capabilities of a framework, without trust users will look elsewhere. This is especially
true of those not trained in software development or computational science.

Developing trust in a framework goes beyond utilizing ``best practices'' for the code developed, it is equally
important that the framework itself is built upon tools that are trusted. For this reason, MOOSE relies on a
well-established code base of libMesh and PETSc.
The libMesh library provides foundational capability for the finite element method and provides interfaces to
leading-edge numerical solution packages such as PETSc.

With these principles in mind, an open source, massively parallel, finite element, multiphysics framework has been conceived.
MOOSE is an on-going project started in 2008 aimed toward a common platform for creation of new multiphysics tools.
This document provides design details pertinent to application developers as well as framework developers.

\subsection{Use Cases}
The MOOSE Framework is targeted at two main groups of actors: Developers and Users. Developers are the main
use case. These are typically students and professionals trained in science and engineering fields with some
level of experience with coding but typically very little formal software development training. The other user
group is Users. Those who intend to use an application built upon the framework without writing any computer
code themselves. Instead they may modify or create input files for driving a simulation, run the application, and
analyze the results. All interactions through MOOSE are primarily through the command-line interface and through
a customizable block-based input file.

\subsection{Software Overview}

\subsubsection{Introduction}
The MOOSE framework itself is composed of a wide range of pluggable systems. Each system is generally composed of a
single or small set of C++ objects intended to be specialized by a Developer to solve a specific problem. To accomplish
this design goal, MOOSE uses several modern object-oriented design patterns. The primary overarching pattern is the
``Factory Pattern''. Users needing to extend MOOSE may inherit from one of MOOSE's systems to providing an implementation
meeting his or her needs. The design of each of these systems is documented on the mooseframework.org wiki in the Tutorial
section. Additionally, up-to-date documentation extracted from the source is maintained on the the mooseframework.org
documentation site after every successful merge to MOOSE's stable branch. After these objects are created, the can be
registered with the framework and used immediately in a MOOSE input file.

\subsubsection{System Architecture}
The MOOSE framework architecture consists of a core and several pluggable systems. The core of MOOSE consists of a number
of key objects responsible for setting up and managing the user-defined objects of a finite element simulation. This core
set of objects has limited extendability and exist for every simulation configuration that the framework is capable of running.

\begin{itemize}
\item MooseApp
\item Problem
\item System(s)
\item Mesh
\item Factory(s)
\item Warehouses
\item Assembly
\item ThreadedComputeLoops
\end{itemize}

The MooseApp is the top-level object used to hold all of the other objects in a simulation. In a normal simulation a single
MooseApp object is created and ``run()''. This object uses it's Factory objects to build user defined objects which are stored
in a series of Warehouse objects and executed. The Finite Element data is stored in the Systems and Assembly object while the
domain information (the Mesh) is stored in the Mesh object. A series of threaded loops are used to run parallel calculations
on the objects created and stored within the warehouses.

\subsection{Pluggable Systems}
MOOSE's pluggable systems are documented on the mooseframework.org wiki. Each of these systems has set of defined polymorphic
interfaces and are designed to accomplish a specific task within the simulation. The design of these systems is fluid and is
managed through agile methods and ticket request system on the Github.org website.

\end{document}

%%  LocalWords:  Multiphysics MOOSE's Jacobian Krylov JFNK Multiapp
%%  LocalWords:  POSIX ASME NQA LWP IEC IEEE INL OOP QA GCC Galerkin
%%  LocalWords:  discretized multi MPI pthreads TBB OpenMP LGPL PDEs
%%  LocalWords:  API APIs composable libMesh PETSc multiphysics wiki
%%  LocalWords:  pluggable mooseframework MooseApp Github
%%  LocalWords:  ThreadedComputeLoops
