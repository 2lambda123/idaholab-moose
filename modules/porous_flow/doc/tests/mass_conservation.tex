\documentclass[]{scrreprt}
\usepackage{amsmath,amsfonts,graphicx}
\usepackage{multirow}
\usepackage{pslatex}
\usepackage{tabularx}
\usepackage{comment}
\usepackage{xspace}
\usepackage{array}

\usepackage{hyperref}

\usepackage{caption}
\DeclareCaptionFont{white}{\color{white}}
\DeclareCaptionFormat{listing}{\colorbox{gray}{\parbox{\textwidth}{#1#2#3}}}

\graphicspath{
{figures/}
}

\def\species{\mathrm{sp}}
\def\phase{\mathrm{ph}}
\def\massfrac{\chi}
\def\flux{\mathbf{F}}
\def\darcyvel{\mathbf{v}}
\def\energydens{\mathcal{E}}
\def\d{\mathrm{d}}

\newcommand{\uo}{\mbox{UO\textsubscript{2}}\xspace}

\setcounter{secnumdepth}{3}


\begin{document}


\title{Mass-Conservation Tests}
\author{CSIRO}
\maketitle

\tableofcontents

\chapter{PorousFlowFluidMass postprocessor}

\section{Single-phase, single-component}
\label{1phase1comp.sec}

The total fluid mass of species $\species$ within a volume $V$ is
\begin{equation}
\int_{V} \phi\sum_{\phase}\rho_{\phase} S_{\phase}\massfrac_{\phase}^{\species} \ .
\end{equation}
It must be checked that MOOSE calculates this correctly in order that
mass-balances be correct, and also because this quantity is used in a
number of other tests

A 1D model with $-1\leq x \leq 1$, and with three elements of size 1 is
created with the following properties:
\begin{center}
\begin{tabular}{|ll|}
\hline
Constant fluid bulk modulus & 1\,Pa \\
Fluid density at zero pressure & 1\,kg.m$^{-3}$ \\
Van Genuchten $m$ & 0.5 \\
Van Genuchten $\alpha$ & 1\,Pa$^{-1}$ \\
Porosity & 0.1 \\
\hline
\end{tabular}
\end{center}
The porepressure is set at $P=x$.

Recall that in PorousFlow, mass is lumped to the nodes.  Therefore,
the integral above is evaluated at the nodes, and a sum of the results
is outputted as the PorousFlowFluidMass postprocessor.
Using the properties given above, this yields:
\begin{center}
\begin{tabular}{|ccccc|}
\hline
$x$ & $p$ & Density & Saturation & Nodal mass \\
\hline
-1 & -1 & 0.367879441 & 0.707106781 & 0.008671002 \\
-0.333333333 & -0.333333333 & 0.716531311 & 0.948683298 & 0.02265871 \\
-0.333333333 & -0.333333333 & 0.716531311 & 0.948683298 & 0.02265871 \\
0.333333333 & 0.333333333 & 1.395612425 & 1 & 0.046520414 \\
0.333333333 & 0.333333333 & 1.395612425 & 1 & 0.046520414 \\
1& 1 & 2.718281828 & 1 & 0.090609394 \\
\hline
 & & & Total & 0.237638643 \\
\hline
\end{tabular}
\end{center}
MOOSE also gives the total mass as 0.237638643\,kg.  This test is part of
the automatic test suite that is run every time the code is updated.

\newpage

\section{Single-phase, two-components}

The same test as Section~\ref{1phase1comp.sec} is run but with two
components.  The mass fraction is fixed at
\begin{equation}
\massfrac_{\phase=0}^{\species=0} = x^{2} \ .
\end{equation}

\begin{center}
\begin{tabular}{|ccccccc|}
\hline
$x$ & $p$ & Density & Saturation & $\massfrac_{\phase=0}^{\species=0}$
& Nodal mass$_{\species=0}$ & Nodal mass$_{\species=1}$ \\
\hline
-1 & -1 & 0.367879441 & 0.707106781 & 1 & 0.008671 & 0 \\
-0.333333333 & -0.333333333 & 0.716531311 & 0.948683298 & 0.111111 &
0.00251763 & 0.02014108 \\
-0.333333333 & -0.333333333 & 0.716531311 & 0.948683298 & 0.111111 &
0.00251763 & 0.02014108 \\
0.333333333 & 0.333333333 & 1.395612425 & 1 & 0.111111 & 0.00516893 &
0.04135148 \\
0.333333333 & 0.333333333 & 1.395612425 & 1 & 0.111111 & 0.00516893 &
0.04135148 \\
1& 1 & 2.718281828 & 1 & 1 & 0.09060939 & 0 \\
\hline
 & & & & Total & 0.11465353 & 0.12298511 \\
\hline
\end{tabular}
\end{center}

MOOSE produces the expected answer.


\end{document}

