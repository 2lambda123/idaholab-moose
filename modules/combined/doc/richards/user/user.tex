\documentclass[]{scrreprt}
\usepackage{amsmath,amsfonts,graphicx}
\usepackage{multirow}
\usepackage{pslatex}
\usepackage{tabularx}
\usepackage{comment}
\usepackage{xspace}
\usepackage{array}

\usepackage{hyperref}

\usepackage{caption}
\DeclareCaptionFont{white}{\color{white}}
\DeclareCaptionFormat{listing}{\colorbox{gray}{\parbox{\textwidth}{#1#2#3}}}

\graphicspath{
{figures/}
}

\newcommand{\uo}{\mbox{UO\textsubscript{2}}\xspace}

\setcounter{secnumdepth}{3}


\begin{document}


\title{Richards Examples}
\author{Andy Wilkins \\
CSIRO}
\maketitle

\tableofcontents

%%%
\chapter{Introduction}
%%%

The Richards' equation describes slow fluid flow through a porous
medium.  This document outlines input-file examples for the Richards
MOOSE code, drawing mainly upon the test suite.  There are two other
accompanying documents: (1) The theoretical and numerical foundations
of the code, which also describes the notation used throughout this
document; (2) The test suite, which describes the benchmark tests used
to validate the code. 


\chapter{The examples}

Each example is located in the {\em test} directory, which has path
\begin{verbatim}
<install_dir>/trunk/elk/tests/richards
\end{verbatim}
or the {\em user} directory, which has path
\begin{verbatim}
<install_dir>/trunk/elk/doc/richards/user
\end{verbatim}

\section{Convergence and convergence criteria}

As a general rule, the formulation of multiphase flow implemented in
MOOSE is quite suitable to solve many different problems.  However,
there are some situations where the implementation is not optimal,
such as the tracking of fronts.  MOOSE may converge in these
instances, but may take unacceptably short time steps, or it may not
even converge at all.  In these cases it is probably best to either
modify the problem to something more suitable, or use another code.

For example, in simulations with sharp fronts, the user should ask
whether it is truly necessary to accurately track the sharp front, or
whether similar results could be taken by smoothing the fronts by
using slightly different initial conditions, and/or by modifying the
effective-saturation relationships, by making the van Genuchten
parameter $\alpha$ smaller, for instance.  If tracking sharp fronts is
vital to the problem then a code specifically designed to solve such
problems will do it better than MOOSE, or perhaps MOOSE could be used
with an ALE front-tracking algorithm.

Below I list some general pointers that may help with problems that
MOOSE is struggling with.
\begin{itemize}
\item Effective saturation curves that are too ``flat'' are not good.
  For example, the van Genuchten parameter $m=0.6$ almost always gives
  better convergence than $m=0.9$.
\item Effective saturation curves that are too ``low'' are not good.
  For example, the van Genuchten parameter $\alpha=10^{-6}$\,Pa$^{-1}$
  almost always gives better convergence than
  $\alpha=10^{-3}$\,Pa$^{-1}$.  Compare, for instance, the tests {\tt
    buckley\_leverett/bl21.i} and {\tt buckley\_leverett/bl22.i}.
\item Any discontinuities in the effective saturation, or its
  derivative, are bad.  I suggest using van Genuchten parameter
  $\alpha\geq 0.5$ for problems with both saturated and unsaturated zones.
\item Highly nonlinear relative permeability curves make convergence
  difficult in some cases.  For instance, a ``power'' relative
  permeability curve with $n=20$ is much worse numerically than with
  $n=2$.  See if you can reduce the nonlinearity in your curve.
\item Any discontinuities in the relative permeability, or its derivative,
  are bad.  For most curves coded into MOOSE this is not an issue, but
  I recommend the {\tt RichardsRelPermVG1} curve over {\tt
    RichardsRelPermVG}, since the former is smooth around
  $S_{\mathrm{eff}}=1$.  See {\tt recharge\_discharge/rd01.i} in the
  tests directory for an example of this.   I also suggest using the
  ``power'' covers over the ``VG'' curves.
\end{itemize}

\section{Two-phase, almost saturated}

If a two-phase model has regions that are fully saturated with the
``1'' phase (typically this is water), then the residual for the ``2''
phase is zero.  This means the ``2''-phase pressure will not change in
those regions, potentially violating $P_{1}\leq P_{2}$.  If the ``2''
phase subsequently infiltrates to these regions, an initially crazy
$P_{2}$ might affect the results.  This sometimes also holds for
almost-saturated situations, depending on the exact simulation.

In these cases it is useful to add a penalty term to the residual to
ensure that $P_{1}\leq P_{2}$.  An example can be found in the tests
directory {\tt pressure\_pulse/pp22.i}.  The choice of the $a$
parameter is sometimes difficult: too big and the penalty term
dominates the Darcy flow; too small and the penalty term does
nothing.  In both cases, convergence is poor as the penalty term
switches on and off during the Newton-Raphson procedure.  The
documentation for {\tt RichardsPPenalty} describes how to set $a$ (run
MOOSE with a {\tt -\,-dump} flag).

The penalty term should {\em not} be used unless absolutely necessary
as it will lead to poorer convergence characteristics.  In many cases
it is not necessary.

\section{An excavation}

In the test directory {\tt excav/ex01.i} and {\tt excav/ex02.i}
contains a single excavation, and the associated mass flux and mass
balance.

\section{Bounding porepressure}

Occasionally it might be useful to bound porepressure.  The test {\tt
  buckley\_leverett/bl22.i} has ``bounds'' that do this.  Note that:
\begin{itemize}
\item The convergence is likely to be much slower when using bounds
\item The {\tt -snes\_type} must be set to {\tt vinewtonssls} (see the
  {\tt [Preconditioning]} block of the aforementioned test.
\item The command line must contain the argument {\tt --use-petsc-dm}.
\end{itemize}








\end{document}

