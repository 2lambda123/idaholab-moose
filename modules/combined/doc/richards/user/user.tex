\documentclass[]{scrreprt}
\usepackage{amsmath,amsfonts,graphicx}
\usepackage{multirow}
\usepackage{pslatex}
\usepackage{tabularx}
\usepackage{comment}
\usepackage{xspace}
\usepackage{array}

\usepackage{hyperref}

\usepackage{caption}
\DeclareCaptionFont{white}{\color{white}}
\DeclareCaptionFormat{listing}{\colorbox{gray}{\parbox{\textwidth}{#1#2#3}}}

\graphicspath{
{figures/}
}

\newcommand{\uo}{\mbox{UO\textsubscript{2}}\xspace}

\setcounter{secnumdepth}{3}


\begin{document}


\title{Richards Examples}
\author{Andy Wilkins \\
CSIRO}
\maketitle

\tableofcontents

%%%
\chapter{Introduction}
%%%

The Richards' equation describes slow fluid flow through a porous
medium.  This document outlines input-file examples for the Richards
MOOSE code, drawing mainly upon the test suite.  There are two other
accompanying documents: (1) The theoretical and numerical foundations
of the code, which also describes the notation used throughout this
document; (2) The test suite, which describes the benchmark tests used
to validate the code.


\chapter{The examples}

Each example is located in the {\em test} directory, which has path
\begin{verbatim}
<install_dir>/trunk/elk/tests/richards
\end{verbatim}
or the {\em user} directory, which has path
\begin{verbatim}
<install_dir>/trunk/elk/doc/richards/user
\end{verbatim}

\section{Convergence and convergence criteria}

As a general rule, the formulation of multiphase flow implemented in
MOOSE is quite suitable to solve many different problems.  However,
there are some situations where the implementation is not optimal,
such as the tracking of fronts.  MOOSE may converge in these
instances, but may take unacceptably short time steps, or it may not
even converge at all.  In these cases it is probably best to either
modify the problem to something more suitable, or use another code.

For example, in simulations with sharp fronts, the user should ask
whether it is truly necessary to accurately track the sharp front, or
whether similar results could be taken by smoothing the fronts by
using slightly different initial conditions, and/or by modifying the
effective-saturation relationships, by making the van Genuchten
parameter $\alpha$ smaller, for instance.  If tracking sharp fronts is
vital to the problem then a code specifically designed to solve such
problems will do it better than MOOSE, or perhaps MOOSE could be used
with an ALE front-tracking algorithm.

Below I list some general pointers that may help with problems that
MOOSE is struggling with.
\begin{itemize}
\item Effective saturation curves that are too ``flat'' are not good.
  For example, the van Genuchten parameter $m=0.6$ almost always gives
  better convergence than $m=0.9$.
\item Effective saturation curves that are too ``low'' are not good.
  For example, the van Genuchten parameter $\alpha=10^{-6}$\,Pa$^{-1}$
  almost always gives better convergence than
  $\alpha=10^{-3}$\,Pa$^{-1}$.  Compare, for instance, the tests {\tt
    buckley\_leverett/bl21.i} and {\tt buckley\_leverett/bl22.i}.
\item Any discontinuities in the effective saturation, or its
  derivative, are bad.  I suggest using van Genuchten parameter
  $\alpha\geq 0.5$ for problems with both saturated and unsaturated zones.
\item Highly nonlinear relative permeability curves make convergence
  difficult in some cases.  For instance, a ``power'' relative
  permeability curve with $n=20$ is much worse numerically than with
  $n=2$.  See if you can reduce the nonlinearity in your curve.
\item Any discontinuities in the relative permeability, or its derivative,
  are bad.  For most curves coded into MOOSE this is not an issue, but
  I recommend the {\tt RichardsRelPermVG1} curve over {\tt
    RichardsRelPermVG}, since the former is smooth around
  $S_{\mathrm{eff}}=1$.  See {\tt recharge\_discharge/rd01.i} in the
  tests directory for an example of this.   I also suggest using the
  ``power'' covers over the ``VG'' curves.
\end{itemize}

Choosing reasonable convergence criteria is very important.  The
Theory Manual contains a section that explains the {\em minimum}
residual that a user can expect to obtain in a model.  However, this
minimum is usually much smaller than what is important from a
practical point of view.  If a tiny residual is chosen, MOOSE will
spend most of its time changing pressure values by tiny amounts as it
attempts to converge to the tiny residual, and this is a waste of
compute time.   This problem may be amplified if adaptive time-stepping is
used since MOOSE doesn't realise that most of the compute time is
spent ``doing nothing'', so keeps the timestep very small.  So, here
are some guidelines for choosing an appropriate tolerance on the
residual.
\begin{enumerate}
\item Determine an appopriate tolerance on what you mean by
  ``steadystate''.  For instance, in a single-phase simulation with
  reasonably large constant fluid bulk modulus, and gravity acting in
  the $-z$ direction, the steadystate solution is $P = -\rho_{0}gz$
  (up to a constant).  In the case of water, this reads $P=-10000z$.
  Instead of this, suppose you would be happy to say the model is at
  steadystate if $P = -(\rho_{0} g + \epsilon)z$.  For instance, for water,
  $\epsilon=1$\,Pa.m$^{-1}$ might be suitable in your problem.  Then recall
  that the residual is just
\begin{equation}
R = \left|\int
\nabla_{i}\left(\frac{\kappa_{ij}\kappa_{\mathrm{rel}}\rho}{\mu}(\nabla_{j}P
+ \rho g_{j}) \right) \right|
\label{eqn.res.int}
\end{equation}
Evaluate this for your ``steadystate'' solution.  For instance, in the
case of water just quoted, $R = V|\kappa|\rho_{0}/\mu\epsilon =
V|\kappa|\times 10^{6}\epsilon$, where: $V$ is the volume of the
finite-element mesh, and I have inserted standard values for
$\rho_{0}$ and $\mu$.
\item In the previous step, an appropriate tolernace on the residual
  was given as $V|\kappa|\rho_{0}\epsilon/\mu$.  However, this is often too
  large because of the factor of $V$.  The previous step assumed that
  the solution was incorrect by a factor, $\epsilon$, which is constant
  over the entire mesh.   More commonly, there is a small region of
  the mesh where most of the interesting dynamics occurs, and the
  remainder of the mesh exists just to provide reasonable boundary
  conditions for this ``interesting'' region.  The residual in the
  ``boring'' region can be thought of as virtually zero, while the
  residual in the ``interesting'' region is
  $V_{\mathrm{interesting}}|\kappa|\rho_{0}\epsilon/\mu$.  This is smaller
  than the residual in the previous step, so provides a tighter
  tolerance for MOOSE to strive towards.
\item In the previous steps, I've implicitly assumed $\kappa$ is
  constant, $\rho$ is virtually constant at $\rho_{0}$, only a
  single-phase situation, etc.  In many cases these assumptions are
  not valid, so the integral of Eqn~(\ref{eqn.res.int}) cannot be done
  as trivially as in the previous steps.  In these cases, I simply
  suggest to build a model with initial conditions like $P =
  -(\rho_{0} g + \epsilon)z$, and just see what the initial residual is.
  That will give you an idea of how big a reasonable residual
  tolerance should be.
\end{enumerate}


\section{Two-phase, almost saturated}

If a two-phase model has regions that are fully saturated with the
``1'' phase (typically this is water), then the residual for the ``2''
phase is zero.  This means the ``2''-phase pressure will not change in
those regions, potentially violating $P_{1}\leq P_{2}$.  If the ``2''
phase subsequently infiltrates to these regions, an initially crazy
$P_{2}$ might affect the results.  This sometimes also holds for
almost-saturated situations, depending on the exact simulation.

In these cases it is useful to add a penalty term to the residual to
ensure that $P_{1}\leq P_{2}$.  An example can be found in the tests
directory {\tt pressure\_pulse/pp22.i}.  The choice of the $a$
parameter is sometimes difficult: too big and the penalty term
dominates the Darcy flow; too small and the penalty term does
nothing.  In both cases, convergence is poor as the penalty term
switches on and off during the Newton-Raphson procedure.  The
documentation for {\tt RichardsPPenalty} describes how to set $a$ (run
MOOSE with a {\tt -\,-dump} flag).

The penalty term should {\em not} be used unless absolutely necessary
as it will lead to poorer convergence characteristics.  In many cases
it is not necessary.

\section{An excavation}

In the test directory {\tt excav/ex01.i} and {\tt excav/ex02.i}
contains a single excavation, and the associated mass flux and mass
balance.

\section{Bounding porepressure}

Occasionally it might be useful to bound porepressure.  The test {\tt
  buckley\_leverett/bl22.i} has ``bounds'' that do this.  Note that:
\begin{itemize}
\item The convergence is likely to be much slower when using bounds
\item The {\tt -snes\_type} must be set to {\tt vinewtonssls} (see the
  {\tt [Preconditioning]} block of the aforementioned test.
\item The command line must contain the argument {\tt --use-petsc-dm}.
\end{itemize}








\end{document}

