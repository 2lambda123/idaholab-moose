\documentclass[]{scrreprt}
\usepackage{amsmath,amsfonts,graphicx}
\usepackage{multirow}
\usepackage{pslatex}
\usepackage{tabularx}
\usepackage{comment}
\usepackage{xspace}
\usepackage{array}

\usepackage{hyperref}

\usepackage{caption}
\DeclareCaptionFont{white}{\color{white}}
\DeclareCaptionFormat{listing}{\colorbox{gray}{\parbox{\textwidth}{#1#2#3}}}

\graphicspath{
{figures/}
}

\newcommand{\uo}{\mbox{UO\textsubscript{2}}\xspace}

\setcounter{secnumdepth}{3}


\begin{document}


\title{Tensor Mechanics Theory Manual}
\author{Andy and Ioannis}
\maketitle

\tableofcontents

%%%
\chapter{Definitions and Equations}
%%%

In a 3D Cosserat continuum there are 6 degrees of freedom.  These are
three translations, $u_{i}$, ($i=1,\ 2\, 3$), and Cosserat rotations,
$\omega^{c}_{i}$ ($i=1,\ 2\, 3$).   From these form the strain tensor
\begin{equation}
\gamma_{ij} = \nabla_{j}u_{i} + \epsilon_{ijk}\omega^{c}_{k} \ ,
\end{equation}
and the curvature
\begin{equation}
\kappa_{ij} = \nabla_{j}\omega^{c}_{i}
\end{equation}

The consitutive laws are
\begin{eqnarray}
\tau_{ij} & = & C_{ijkl}\gamma_{kl} \ , \\
M_{ij} & = & B_{ijkl}\kappa_{kl} \ .
\end{eqnarray}
Here $\tau_{ij}$ is the stress (it is not symmetric), and $M_{ij}$ is
the couple stress.

The equations of motion are
\begin{eqnarray}
\rho \ddot{u}_{i} & = & \nabla_{j}\tau_{ij} + b_{i} \ , \\
I_{ij}\ddot{\omega^{c}}_{i} & = & \nabla_{j}M){ij} -
\epsilon_{ijk}tau_{jk} + \tilde{b}_{i} \ .
\end{eqnarray}
Here $\rho$ is the density and $I_{ij}$ is the moment of intertia.




\end{document}

