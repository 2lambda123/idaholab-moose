
\chapter{\label{devel-extension}Devel Extension}
\section{\label{examples}Examples}
\begin{example}[label=markdown-example]{The example command.}

\begin{verbatim}
This is a +test+.
\end{verbatim}

\tcblower
\par This is a \textbf{test}.
\end{example}

\begin{example}[label=markdown-example2]{The example command with block format.}

\begin{verbatim}
First paragraph.

Second paragraph.
\end{verbatim}

\tcblower
\par First paragraph.
\par Second paragraph. 
\end{example}

\section{\label{settings}Settings}
\par The available settings for code blocks are listed in Table~\ref{code-settings}.\par\begin{tabulary}{\textwidth}{LLL}
\toprule
Key &Default &Description \\
\midrule
style &None &The style settings that are passed to rendered HTML tag. \\
class &None &The class settings to be passed to rendered HTML tag. \\
id &None &The class settings to be passed to the rendered tag. \\
language &text &The code language to use for highlighting. \\
\bottomrule
\end{tabulary}

\begin{table}
\center\caption{\label{code-settings}Settings for code blocks.}\par\begin{tabulary}{\textwidth}{LLL}
\toprule
Key &Default &Description \\
\midrule
style &None &The style settings that are passed to rendered HTML tag. \\
class &None &The class settings to be passed to rendered HTML tag. \\
id &None &The class settings to be passed to the rendered tag. \\
language &text &The code language to use for highlighting. \\
\bottomrule
\end{tabulary}

\end{table}
