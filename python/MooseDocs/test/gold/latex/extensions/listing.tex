
\chapter{\label{listing-extension}Listing Extension}
\section{\label{test-numbering}Test Numbering}
\begin{lstlisting}[label=one,caption=\mbox{}]
One
\end{lstlisting}

\begin{lstlisting}[label=two,caption=\mbox{}]
Two
\end{lstlisting}

\begin{lstlisting}[label=three,caption=\mbox{}]
Three
\end{lstlisting}

\par You can reference listings: Listing~\ref{three}.
\section{\label{test-captions}Test Captions}
\begin{verbatim}
no caption
\end{verbatim}

\begin{lstlisting}[title={the \textbf{caption}}]
caption with inline content
\end{lstlisting}

\begin{lstlisting}[label=four,caption={the \ul{caption}}]
caption with number
\end{lstlisting}

\section{\label{listing-with-space}Listing with space}
\begin{verbatim}
void function();

void anotherFunction();
\end{verbatim}

\section{\label{language}Language}
\begin{verbatim}
void function();
\end{verbatim}

\section{\label{file-listing}File Listing}
\begin{lstlisting}[language=C++,label=diffusion-c,caption=\mbox{}]
#include "Diffusion.h"

registerMooseObject("MooseApp", Diffusion);

template <>
InputParameters
validParams<Diffusion>()
{
  InputParameters params = validParams<Kernel>();
  params.addClassDescription("The Laplacian operator ($-\\nabla \\cdot \\nabla u$), with the weak "
                             "form of $(\\nabla \\phi_i, \\nabla u_h)$.");
  return params;
}

Diffusion::Diffusion(const InputParameters & parameters) : Kernel(parameters) {}

Real
Diffusion::computeQpResidual()
{
  return _grad_u[_qp] * _grad_test[_i][_qp];
}

Real
Diffusion::computeQpJacobian()
{
  return _grad_phi[_j][_qp] * _grad_test[_i][_qp];
}
\end{lstlisting}

\begin{verbatim}
template <>
InputParameters
validParams<Diffusion>()
{
  InputParameters params = validParams<Kernel>();
  params.addClassDescription("The Laplacian operator ($-\\nabla \\cdot \\nabla u$), with the weak "
                             "form of $(\\nabla \\phi_i, \\nabla u_h)$.");
  return params;
}
\end{verbatim}

\begin{verbatim}
xxxxx[Mesh]
xxxxx  type = GeneratedMesh
xxxxx  dim = 2
xxxxx  nx = 10
xxxxx  ny = 10
xxxxx[]
xxxxx
xxxxx[Kernels]
xxxxx  [./diff]
xxxxx    type = Diffusion
xxxxx    variable = u
xxxxx  [../]
xxxxx[]
\end{verbatim}

\begin{verbatim}
[AuxKernels]
    [./diff]
      type = Diffusion
      variable = u
    [../]
[./]
\end{verbatim}
